\documentclass{report}


\usepackage{lmodern}
\usepackage{amsmath,amsfonts}
\usepackage{listings}
\usepackage{color} 
\usepackage{graphicx} 
\usepackage[latin1]{inputenc}
\usepackage[T1]{fontenc}
%\usepackage[francais]{babel}
\usepackage{pdfpages}
\usepackage[top=2cm, bottom=2cm, left=3cm, right=3cm]{geometry}
\usepackage{fancyvrb}
\usepackage{mathrsfs}
\usepackage{listings}
\usepackage{verbatim}
\usepackage{url}


\title{Motion Planning Workshop}
\author{Sourava Prasad Mishra, Amrit Kumar}
\date{April 17 2013}

\begin{document}
\maketitle

\section{Introduction}

	This program is created to illustrate the autonomous path finding from a source to a destination point in a given grid. This problem is solved using two methods and the user will be presented with outputs from both the programs. The program uses a specific data structure to store the workspace and two motion planning schemes namely:
	\begin{enumerate}
	\item A heuristics navigation function
	\item A star algorithm
	\end{enumerate}

	The general working of this motion planning program is as follows:
	\begin{enumerate}
	\item Firstly, an instance of the data structure is created.
	\item Then the user is asked to enter various workspace parameters at run time.
	\item The workspace is initialized along with a check on the source and destination coordinate, for if they are over an obstacle.
	\item A distance matrix is computed using \textsf{wavefront algorithm}[1], to find the distance of all the blocks from the destination.
	\item If exist, an optimal path is calculated from source to destination using the above obtained distance matrix and printed on standard output.
	\item Similarly a path is created using A * algorithm and printed on standard output.
	\end{enumerate}

\section{Data structures}
	Proper storage and programmatic representation of the workspace is an important aspect in this context of motion planning. We are using a \textsf{C} programming language \textsf{structure} to encapsulate the workspace and further typed it with \textsf{typedef}. The name of structure is \textsf{Workshop}, typed as \textsf{workspace} and it contains	
	\begin{itemize}
	\item the dimensions of Workspace as \textsf{integers}.
	\item source and destinations coordinates as of type \textsf{integers}.
	\item two, two dimensional \textsf{arrays} for storing the individual values in the workshop and the distance matrix.
	\item two additional \textsf{arrays} for storing the (x, y) coordinates of the path from source to destination.
	\end{itemize}
	
	The placing of obstacles in the workspace is a random process and around little less then 50 percent of blocks in the workspace containd obstacles. This however can be changed at will, in the function \textsf{input()} of \textsf{util.c} program.

\section{Navigation function}
	Navigation function is an technique used to estimate the path from the source to destination. The algorithm used to implement this navigation function is a breadth first search (BFS) technique traversal algorithm[2]. BFS is a graph search technique and uses basis queue operations while traversing the components. The implementation of navigation function, finds a node which is nearest to the current node and adds it to the path. The space complexity of of a BFS algorithm is at most \textsf{O(|V|+|E|)} times, where V and E are the number of vertices and edges respectively. The reason for this is, during the traversal in worst cases, every vertices will be traversed at most once.


\section{A star algorithm}
	Our final objective was to implement A* search algorithm[3], which as well is fairly straight foreword.

\section{References}
	\begin{enumerate}
	\item Wavefront algorithm, \url{http://www.societyofrobots.com/programming_wavefront.shtml}
	\item Breadth first search traversal algorithm, \url{http://www.personal.kent.edu/~rmuhamma/Algorithms/MyAlgorithms/GraphAlgor/breadthSearch.htm}
	\item A* algorithm \url{http://en.wikipedia.org/wiki/A*_search_algorithm}
	\end{enumerate}

\end{document}

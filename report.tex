\documentclass[12pt]{article}

\pagenumbering{Roman}
\usepackage[T1]{fontenc}
\usepackage[utf8]{inputenc}
\usepackage{textcomp}
\usepackage[english,french]{babel}
\usepackage{etex}
\usepackage[left=1.0in, right=1.0in, top=1.0in, bottom=1.0in]{geometry}
\usepackage{layout}
\usepackage{lmodern}
\usepackage{amsthm}
\usepackage{amsmath}
\usepackage{amsfonts}
\usepackage{amssymb}
\usepackage{textcomp}
\usepackage{graphicx}
\usepackage{hyperref}
\usepackage[all]{hypcap}
\usepackage{tabularx}
\usepackage[usenames,dvipsnames]{xcolor}
\usepackage{listings}
\usepackage{algorithm2e}
\hypersetup{
  bookmarks=true,         % show bookmarks bar?
  unicode=false,          % non-Latin characters in Acrobat’s bookmarks
  pdftoolbar=true,        % show Acrobat’s toolbar?
  pdfmenubar=true,        % show Acrobat’s menu?
  pdffitwindow=false,     % window fit to page when opened
  pdfstartview={FitH},    % fits the width of the page to the window
  pdftitle={My title},    % title
  pdfauthor={Author},     % author
  pdfsubject={Subject},   % subject of the document
  pdfcreator={Creator},   % creator of the document
  pdfproducer={Producer}, % producer of the document
  pdfkeywords={keyword1} {key2} {key3}, % list of keywords
  pdfnewwindow=true,      % links in new window
  colorlinks=true,       % false: boxed links; true: colored links
  linkcolor=cyan,          % color of internal links (change box color with linkbordercolor)
  citecolor=green,        % color of links to bibliography
  filecolor=magenta,      % color of file links
  urlcolor=cyan           % color of external links
}



\title{Motion Planning Workshop \\ Introduction to Perception and Robotics}
\author{\textsc{Sourava Prasad Mishra} \& \textsc{Amrit Kumar}
  \\ \textsc{Mosig} \\ \textsc{Ensimag-Ujf}  }
\date{April 17 2013}

\begin{document}
\maketitle

\section{Introduction}

	The objective of this project is to implement two autonomous path
  finding algorithms from a source to a destination point in a
  discrete environment with obstacles. As demanded, we present the
  following two solutions detailed in the next sections:

  \begin{enumerate}
	\item Discrete numeric navigation function with gradient descent
	\item A* algorithm
	\end{enumerate}

The implementation provides a user with results from both the above
cited methods. A specific data structure is conceived to store the
workspace upon which the two motion planning schemes have been built.

The general working of the presented motion planning program is as follows:
	\begin{enumerate}
	\item Firstly, an instance of the data structure is created.
	\item Then the user is asked to enter various workspace parameters
    at run-time. 
	\item The workspace is initialized (with random obstacles) along with a check on the source and destination coordinates, for if they are over an obstacle.
	\item The navigation function is used to compute the distance
    matrix using \textsf{wavefront algorithm} [\ref{item:1}]. The distance matrix
    stores the distance of all the cells from the destination.
	\item An optimal path, if exists is calculated from source to
    destination using the previously obtained distance matrix and printed
    on the standard output.
	\item Similarly a path is found using A* algorithm and printed on the standard output.
	\end{enumerate}

\section{Data Structure}
	Proper storage and programmatic representation of the workspace is
  an important aspect in the context of motion planning. We are using
  a \textsf{C} programming language \textsf{struct} [\autoref{struct}] to encapsulate the
  workspace and further typed with \textsf{typedef}. The structure
  is named \textsf{Workspace}, typed as \textsf{workspace} and it
  contains :	
	\begin{itemize}
	\item the dimensions of Workspace as \textsf{integers}.
	\item source and destination coordinates are of type \textsf{integers}.
	\item $2$ two-dimensional \textsf{arrays}, for storing the cell-type of the workspace and the distance matrix.
    
	The convention used to represent workspace is : $0$ for empty
  cells and $-1$ for obstacles. The placing
  of obstacles in the workspace is a random process and consists
  {of 50\% of all the cells} in the
  workspace. However this can be changed if desired, in the function \textsf{input()} of \textsf{util.c} program.

	\item two additional \textsf{arrays} for storing the $(x, y)$
    coordinates of the cells forming the path from source to
    destination. \\
	\end{itemize}
  
  
  \lstset{                                    % line wrapping on
  language=C,
  frame=lines,
  captionpos=b
 }


\renewcommand{\lstlistingname}{Code}
	
	\begin{lstlisting}[caption=Workspace struct, label=struct]
	struct  Workspace {
		int nbrows, nbcolumns, sx, sy, dx, dy;
		int **grid, **distance;
		int *pathx, *pathy;
	};
	typedef struct Workspace workspace;
	\end{lstlisting}
	

\section{Navigation Function \& Gradient Descent}
	The discrete navigation function is used to obtain the distance matrix previously discussed. Breadth first search (BFS) traversal
  algorithm [\ref{item:2}] is used for the purpose. {The space complexity of BFS is
    \textsf{O(|V|+|E|)},} where V and E are set of vertices and edges respectively. 

Once the distance matrix is built, gradient descent
  algorithm is applied to find the nearest cell to the current cell
  and is added to the path. The algorithm starts with the source cell and terminates when the current cell becomes the destination cell. 

\subsection{Calculating Distance Matrix}

 The iterative version of the wavefront algorithm is implemented
 which resembles BFS. This is done in the \textsf{void
   computeDistance(workspace* w)} function in \textsf{navigation.c}
 program. The pseudo-code is presented in [\autoref{Code:algo}]: \\


\restylealgo{algoruled}
\linesnumbered
\begin{algorithm}[H]
 \SetVline
 \KwData{workspace:=(grid, distance, source, destination)}
 \KwResult{Computes the distance matrix }
 Queue $Q:=\phi$\;
 Enqueue($Q$, destination)\;
 distance[destination] $\leftarrow$ $0$\;
 \tcc{mark destination as visited}
 grid[destination]=1\;
 \While{Q is not empty}{
  $u$ $\leftarrow$ Dequeue($Q$) \;
  \For{$v$ adjacent to $u$, free and not visited}
  {
    Enqueue($Q$,$v$)\;
    \tcc{mark $v$ as visited}
    grid[$v$]=$1$\; 
    distance[$v$] $\leftarrow$ distance[$u$]+$1$\;
}
 }
 \caption{Compute Distance Matrix \label{Code:algo}}
\end{algorithm}



\section{A* Algorithm}
	Our final objective was to implement A* search algorithm
  [\ref{item:3}], which is a fairly straight-forward popular
  algorithm. The usual \textit{Manhattan distance} is used to estimate
  the heuristic distance.
	
\section{Execution of source code}

	The motion planning workshop code which is submitted along with this
  project have been tested on \textsf{ensibm} server at ENSIMAG. The
  program, however can be executed on any standard Linux terminal by
  typing \textsf{'./motionplanning'}. 

A sample run:
	\begin{verbatim}
 machine@machine:~$./motionplanning
 Enter the size of the grid:5
5

 Enter source coordinate(for x, y >= 0):0
2

 Enter destination coordinate(for x, y >= 0):2
2
The destination or the source is at an obstacle
The source is (0,2) 
The destination is (2,2) 

Displaying the workspace:
 0	-1	 0	 0	 0	
 0	 0	-1	 0	-1	
 0	-1	-1	-1	 0	
-1	 0	-1	 0	 0	
 0	 0	 0	 0	-1	
	\end{verbatim}
	During the execution of the program, placing of obstacles happen
  at random. In the above case, the destination happens to be over an
  obstacle and hence, the program terminates after displaying the
  workspace. In such a case, the user can choose to re-run with a
  different set of values for the \textsf{workspace}.

A sample program output when the source and destination are not over obstacles:
	
	\begin{verbatim}
 machine@machine:~$./motionplanning
 Enter the size of the grid:7
8

 Enter source coordinate(for x, y >= 0):3
3

 Enter destination coordinate(for x, y >= 0):2
2
The source is (3,3) 
The destination is (2,2) 

Displaying the workspace:
 0	 0	-1	 0	 0	-1	-1	-1	
 0	 0	 0	 0	 0	 0	 0	-1	
 0	 0	 0	-1	-1	 0	-1	 0	
-1	 0	 0	 0	 0	-1	 0	 0	
-1	 0	 0	 0	 0	-1	 0	 0	
 0	 0	-1	 0	-1	-1	 0	 0	
-1	-1	 0	 0	-1	 0	-1	-1	

After applying the wavefront algorithm
-----------------------------------------
Displaying the distance matrix:
 4	 3	-1	 3	 4	-1	-1	-1	
 3	 2	 1	 2	 3	 4	 5	-1	
 2	 1	 0	-1	-1	 5	-1	 0	
-1	 2	 1	 2	 3	-1	 0	 0	
-1	 3	 2	 3	 4	-1	 0	 0	
 5	 4	-1	 4	-1	-1	 0	 0	
-1	-1	 6	 5	-1	 0	-1	-1	

Displaying the path found from source to destination 
Starting from -> (3, 3, [2]) -> (3, 2, [1]) -> (2, 2, [0]) -> Goal Reached !!

Result of A* algo
Printing the path now in reveser order
(2,2) -> (3,2) -> (3,3)
	\end{verbatim}
  The zeros except at the destination cell indicate that the cells are
not reachable.

	We provide here three more \textit{interesting} test cases, which
  convinced us of the correctness of the implementation:
	
		
	\begin{verbatim}	
	TEST CASE 1
	-----------
machine@machine:~$./motionplanning
The source is (0,1) 
The destination is (2,2) 

Displaying the workspace:
 0	 0	 0	 0	 0	
 0	-1	-1	-1	 0	
 0	-1	 0	 0	 0	
 0	-1	-1	-1	 0	
 0	 0	 0	 0	 0	

After applying the wavefront algorithm
-----------------------------------------
Displaying the distance matrix:
 8	 7	 6	 5	 4	
 9	-1	-1	-1	 3	
 10	-1	 0	 1	 2	
 9	-1	-1	-1	 3	
 8	 7	 6	 5	 4	

Displaying the path found from source to destination 
Starting from -> (0, 1, [7]) -> (0, 2, [6]) -> (0, 3, [5]) -> (0, 4, [4]) -> (1, 4, [3]) -> (2, 4, [2]) -> (2, 3, [1]) -> (2, 2, [0]) -> Goal Reached !!

Result of A* algo
Printing the path now in reveser order
(2,2) -> (2,3) -> (2,4) -> (1,4) -> (0,4) -> (0,3) -> (0,2) -> (0,1)	
	\end{verbatim}
	


\begin{verbatim}	
	TEST CASE 2
	-----------
machine@machine:~$./motionplanning
The source is (0,0) 
The destination is (2,2) 

Displaying the workspace:
 0	-1	 0	
 0	 0	 0	
 0	-1	 0	
 0	-1	 0	

After applying the wavefront algorithm
-----------------------------------------
Displaying the distance matrix:
 4	-1	 2	
 3	 2	 1	
 4	-1	 0	
 5	-1	 1	

Displaying the path found from source to destination 
Starting from -> (0, 0, [4]) -> (1, 0, [3]) -> (1, 1, [2]) -> (1, 2, [1]) -> (2, 2, [0]) -> Goal Reached !!

Result of A* algo
Printing the path now in reveser order
(2,2) -> (1,2) -> (1,1) -> (1,0) -> (0,0)
	\end{verbatim}



		
	\begin{verbatim}
	TEST CASE 3
	-----------	
machine@machine:~$./motionplanning
The source is (0,1) 
The destination is (2,2) 

Displaying the workspace:
 0	 0	 0	 0	 0	
 0	-1	-1	-1	 0	
 0	-1	 0	-1	 0	
 0	-1	-1	-1	 0	
 0	 0	 0	 0	 0	

After applying the wavefront algorithm
-----------------------------------------
Displaying the distance matrix:
 0	 0	 0	 0	 0	
 0	-1	-1	-1	 0	
 0	-1	 0	-1	 0	
 0	-1	-1	-1	 0	
 0	 0	 0	 0	 0	

The source is (0,1) 
The destination is (2,2) 

Displaying the workspace:
 0	 0	 0	 0	 0	
 0	-1	-1	-1	 0	
 0	-1	 0	-1	 0	
 0	-1	-1	-1	 0	
 0	 0	 0	 0	 0	

Path could not be found
Result of A* algo
Search failed !!	
	\end{verbatim}
	


	

\section{References}
	\begin{enumerate}
	\item \label{item:1}Wavefront algorithm, \href{http://www.societyofrobots.com/programming_wavefront.shtml}{\nolinkurl{http://www.societyofrobots.com/programming_wavefront.shtml}}
	\item \label{item:2} Breadth first search traversal algorithm, \href{http://www.personal.kent.edu/~rmuhamma/Algorithms/MyAlgorithms/GraphAlgor/breadthSearch.htm}{\nolinkurl{http://www.personal.kent.edu/~rmuhamma/Algorithms/MyAlgorithms/GraphAlgor/breadthSearch.htm}}
	\item \label{item:3} A* algorithm \href{http://en.wikipedia.org/wiki/A*_search_algorithm}{\nolinkurl{http://en.wikipedia.org/wiki/A*_search_algorithm}}
	\end{enumerate}

\end{document}
